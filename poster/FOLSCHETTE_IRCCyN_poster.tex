\documentclass[final,t]{beamer} % beamer 3.10: do NOT use option hyperref={pdfpagelabels=false} !
%\documentclass[final,hyperref={pdfpagelabels=false}]{beamer} % beamer 3.07: get rid of beamer warnings
\usepackage[english]{babel}
\usepackage[utf8]{inputenc}
\usepackage{amsmath,amsthm, amssymb, latexsym}
\usepackage{wasysym}
%\usefonttheme[onlymath]{serif}

\useinnertheme{circles}
\usecolortheme{rose}
\usetheme{jdoc}
\setbeamertemplate{navigation symbols}{}

%\boldmath
\usepackage[orientation=portrait,size=a1,scale=1.4]{beamerposter}                  % e.g. for DIN-A1 poster, with optional grid and debug output
%\usepackage[orientation=portrait,size=a0,scale=1.0,printer=rwth-glossy-uv.df]{beamerposter}   % e.g. for DIN-A0 poster with rwth-glossy-uv printer check
\usepackage{tikz}
\usetikzlibrary[decorations.pathmorphing]
\include{macros-ph}

\definecolor{darkred}{rgb}{0.5,0,0}
\definecolor{darkgreen}{rgb}{0,0.5,0}
\definecolor{darkblue}{rgb}{0,0,0.5}
\definecolor{notsodarkred}{rgb}{0.75,0,0}
\definecolor{notsodarkgreen}{rgb}{0,0.75,0}
\tikzstyle{bigedge}=[->, shorten <=10pt, shorten >=10pt, line width=3mm]
\tikzstyle{tinyedge}=[bigedge, shorten <=0pt, shorten >=7pt, line width=.8mm]
\tikzstyle{smalleredge}=[bigedge, line width=2.2mm]
\tikzstyle{warn}=[rounded corners, draw=red, line width=.07em, align=center, text=darkred]
\tikzstyle{nice}=[rounded corners, draw=green, line width=.07em, align=center, text=darkgreen]

%\newcommand{\warn}{\textcolor{red}{$\bigstar$}~}
%\newcommand{\nice}{\textcolor{green}{$\bigstar$}~}
\newcommand{\warn}{\textcolor{notsodarkred}{$\LARGE\frownie$}~}
\newcommand{\nice}{\textcolor{darkgreen}{$\smiley$}~}

\newcommand{\tval}[1]{\textbf{#1}}



\spec{Informatique}
\labo{IRCCyN}
\equipe{MeForBio}

\title{Presentation of the Process Hitting framework and inference\\of Biological Regulatory Networks with Thomas parameters}
\author{Folschette Maxime}
\photo{Moi-extrait.jpg}
\directeur{Olivier Roux}
\encadrant{Morgan Magnin}
\email{Maxime.Folschette@irccyn.ec-nantes.fr}

\newenvironment{myblock}[1]%
{\section{#1}\begin{block}{\thesection. #1}}%
{\end{block}}
\newenvironment{warnblock}[1]%
{\section{#1}\begin{alertblock}{\thesection. #1}}%
{\end{alertblock}}
\newenvironment{niceblock}[1]%
{\section{#1}\begin{exampleblock}{\thesection. #1}}%
{\end{exampleblock}}

\newlength{\columnheight}
\setlength{\columnheight}{64cm}
\newlength{\adjust}
\setlength{\adjust}{.52cm}



\usepackage{color}
\newcommand{\todo}[1]{\textbf{\textcolor{red}{[[#1]]}}}



\begin{document}
\begin{frame}
\vspace*{-.7cm}
\begin{columns}[T]
\begin{column}[b]{0.48\textwidth}
\parbox[t][\columnheight]{\textwidth}{% must be some better way to set the the height, width and textwidth simultaneously

% Colonne 1

\begin{myblock}{Introduction}
  \begin{itemize}
    \item \tval{Systems biology} = studying and understanding of interactions inside biological systems
    \begin{itemize}
      \item A gene produces a protein
      \item Many proteins activate/inhibit other genes
    \end{itemize}
    \smallskip
    \item Biological systems are \tval{very complex},\\and studying them is time/CPU expensive
    %\item They may over-approximate the desired behavior
  \end{itemize}
  %\todo{image}
\end{myblock}
\vfill

\begin{warnblock}{Studying models}
  \begin{itemize}
    \item Models are often \tval{too large to be studied}
    \smallskip
    \item Usual model-checkers have to compute all states
  \end{itemize}
  %\todo{dessin model-checker}
  \begin{figure}
  \begin{tikzpicture}[line width=.05em]
    \node[ellipse, draw=darkgreen, decoration={coil}, decorate, align=center] (system) {Biological\\system};
    \node[rounded corners, draw=blue, right of=system, node distance=12cm] (model) {Model};
    \node[rounded corners, draw=blue, below of=model, node distance=10cm] (stategraph) {State graph};
    \node[rounded corners, draw=gray, left of=stategraph, node distance=12cm] (properties) {\warn Properties};
    
    \path[bigedge, transparent] (model) edge node[inner sep=0pt] (e) {} (stategraph);
    \node[left of=e, left=.5cm, warn] (out) {Computation time = $\infty$\\Memory size = $\infty$};
    \path[warn, shorten >=-5pt] (out) edge (e);
    
    \path[bigedge, bend left] (system) edge (model);
    \path[bigedge, -, shorten >=0pt] (model) edge (e);
    \path[bigedge, draw=notsodarkred, shorten <=-10pt] (e) edge (stategraph);
    \path[bigedge, bend right, draw=notsodarkred] (stategraph) edge (properties);
    \node[circle, draw=red, fill=black, minimum size=5.5mm, inner sep=0pt] at (e) (ee) {};
  \end{tikzpicture}
  \end{figure}
\end{warnblock}
\vfill

\begin{niceblock}{The Process Hitting framework}
  \begin{center}
    \tval{Process Hitting} (PH)
  \end{center}
  \begin{itemize}
    \item Recent framework well adapted to large-scale models
    \smallskip
    \item Atomistic representation of actions
    \smallskip
    \item Very efficient \tval{reachability approximations} [1]
    \smallskip
    \item Handles models with hundreds of components in $<1$s
  \end{itemize}
  %\todo{dessin PH + approx}
  \begin{figure}
  \begin{tikzpicture}[line width=.05em]
    \node[ellipse, draw=darkgreen, decoration={coil}, decorate, align=center] (system) {Biological\\system};
    \node[rounded corners, draw=blue, right of=system, node distance=12cm] (model) {PH model};
    \node[rounded corners, draw=gray, left of=stategraph, node distance=12cm] (properties) {\nice Properties};
    \node[rounded corners, draw=red, right of=properties, node distance=10cm, align=left, font=\footnotesize] (inconc)
      {\warn Inconclusive\\\phantom{\warn}cases};
    
    \path[transparent] (model) edge node[right, inner sep=0pt] (e) {} (properties);
    \node[left of=e, left=3cm, left=.5cm, nice] (approx) {Approximation\\of the behavior};
    \path[nice] (approx) edge (e);
    \path[tinyedge, draw=notsodarkred] (e) edge (inconc);
    
    \path[bigedge, bend left] (system) edge (model);
    \path[bigedge, -, shorten >=0pt] (model) edge (e);
    \path[smalleredge, shorten <=0pt, draw=darkgreen] (e) edge (properties);
    \node[circle, draw=green, fill=black, minimum size=5.5mm, inner sep=0pt] at (e) (ee) {};
  \end{tikzpicture}
  \end{figure}
\end{niceblock}

}
\end{column}
\begin{column}{0.48\textwidth}
\addtolength{\columnheight}{\adjust}
\parbox[t][\columnheight]{\textwidth}{ % must be some better way to set the the height, width and textwidth simultaneously

% Colonne 2
\vspace*{\adjust}

\begin{niceblock}{Translating PH models to BRN}
  \begin{center}
    \tval{Biological Regulatory Network} (BRN)
  \end{center}
  \begin{itemize}
    \item \tval{Widespread} representation for biological models [2]
    \begin{itemize}
      %\item PH = precise atomistic actions
      %\item BRN = global \& intuitive interactions
      \item Interaction Graph (IG) = interactions between genes
      \item Parameters = evolution of each gene
    \end{itemize}
    \smallskip
    \item Method: \tval{exhaustive search} of interactions [3]
  \end{itemize}
  \begin{centering}
  \begin{figure}
    \includegraphics[scale=2.3,trim=2.5cm 0cm 2.1cm 0cm,clip=true]{ex-inf.pdf}
  \end{figure}
  \end{centering}
\end{niceblock}
\vfill

\begin{myblock}{Work in progress}
  \begin{center}
    PH are not \tval{strictly equivalent} to BRN yet
  \end{center}
  \begin{itemize}
    \item Add priorities between actions into PH
    \smallskip
    \item Find a class of models equivalent to BRNs
    \smallskip
    \item Adapt the \tval{reachability approximations}
  \end{itemize}
\end{myblock}
\vfill

\begin{myblock}{Conclusion}
  \begin{itemize}
    \item Formal translation PH $\rightarrow$ BRN
    \smallskip
    \item Implemented into the Pint library
    \smallskip
    \item Efficient results on big models (up to 40 genes)
  \end{itemize}
  %\todo{Table résultats}
  \bigskip
  \small
  \begin{tabular}{r||c|c||c|c||c|c|}
  \multicolumn{3}{c||}{Model specifications} & \multicolumn{2}{c||}{IG inference} & \multicolumn{2}{c|}{Parameters inference}\\
  \hline
    System & Genes & Actions & $\Delta t$ & Edges & $\Delta t$ & Parameters\\
  \hline
    \textbf{\textbf{egfr}} & \textbf{20} & 399 & \textbf{1s} & 50 & \textbf{1s} & 191\\
  \hline
    \textbf{\textbf{tcrsig}} & \textbf{40} & 301 & \textbf{1s} & 54 & \textbf{1s} & 143\\
  \hline
    \textbf{\textbf{tcrsig}} & \textbf{94} & 1124 & \textbf{13s} & 169 & $\infty$ & $0 / 2.10^9$\\
  \hline
    \textbf{\textbf{egfr}} & \textbf{104} & 2356 & \textbf{4min} & 241 & \textbf{1min 30s} & $1.10^6 / 2.10^6$\\
  \hline
  \end{tabular}
\end{myblock}

}
\end{column}
\end{columns}

\vspace*{2cm}
%\todo{Références}
%[1] Paulevé, Magnin, Roux in \textit{Mathematical Structures in Computer Science}, 2012

%[2] Bernot, Cassez, Comet, Delaplace, Müller, Roux in \textit{Electronic Notes in Theoretical Computer Science}, 2007

%[3] Folschette, Paulevé, Inoue, Magnin, Roux in \textit{Computational Methods in Systems Biology}, 2012

[1] Paulevé, Magnin, Roux in MSCS, 2012
\hfill
[2] Bernot, Cassez, Comet, Delaplace, Müller, Roux in ENTCS, 2007

[3] Folschette, Paulevé, Inoue, Magnin, Roux in CMSB, 2012

\end{frame}
\end{document}
